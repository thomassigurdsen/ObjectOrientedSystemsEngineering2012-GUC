\documentclass{report} % was article, thought i'd try with report...
\usepackage{float}
%\usepackage{fullpage} % 1 inch margins
%\usepackage{harvard} % harvard reference style++
\usepackage[hidelinks]{hyperref} % do not remember entirely what this does
%%\usepackage{minted} % [LINUXONLY] list environment with code highlighting

\title{Object oriented systems engineering \\ IMT3102 \\ Portfolio 1}
\author{Martin Kristian Mellum  \\ Thomas Sigurdsen 090320} % enter student number
\date{\today}

\begin{document}
\maketitle       % make title
%%\section{Abstract} % not sure if needed
%%This section is to be written when all else is done. 100-200 words summarizing.
\tableofcontents % make table of contents
\pagebreak	% and break the page, because pretty

\section{Introduction} % not sure if needed
In this first portfolio we will pick 3 - 4 open-source projects. We are going to analyze each of them from a software engineering point of view. Delve into the differences of infrastructure, communication, openness, financial and social aspects of the open-source development model for these projects. Much of this analysis will be done using chapter 3 of Karl Fogel's Producing Open Source Software\cite{kfposs} as a lense.

The projects we have initially chosen are:
\begin{itemize}
\item Media Player Classic\cite{mpcweb}
\item Ez publish\cite{ezpubweb}
\item OpenDungeons\cite{odweb} % I think we should drop this, pick a new better one, and write up a bit on the curiosa on this one.
\end{itemize}

\section{Project introduction}
\subsection{Media Player Classic}
Media Player Classic (MPC) is a minimalistic media player. They try to keep the old and simple look that the Windows Media Player 6.4 had, which was released in 1999. Although, behind the old look it has most of the features that new media player comes with, including codecs to play new formats. 
This is also what they say about their player on their front page:
\begin{quote}
MPC-HC is an extremely light-weight media player for Windows.
The player supports all common video and audio file formats available for playback.\cite{mpcweb}
\end{quote}
As you can see, it stands MPC-HC, and this is a fork of the original MPC. This "HC" version is the one that their adding new features to, along with fixing bugs and libraries. The original MPC was just intended to be the same Classic Player with no more features, and just fixing bugs and updating libraries, but after the leader, Gabest didn't had more time for the project, the project is inactive. %Or more likely dead
Their page seems also very open, and it's easy to see that it's an open source project. You have easy access to their changelog, bug reports, about, and development wiki. From the development wiki site you can easy find what you expect to find.
 
\subsection{Ez Publish}
Ez Publish is a http based content management system. It is built using mainly php and effectively uses a dual licensing scheme.

Their about reads as follows:
\begin{quote}
eZ Systems is in the business of Web Content Management Solutions and has been since 1999. We maintain a global presence in Europe, Asia and the Americas. 

Our large international business partner network is the foundation of our success in the market. 

Our team of engineers, consultants and partner managers enables successful implementations for our customers and partners.\cite{ezpubwebabout}
\end{quote}
The Ez Publish website\cite{ezpubweb} appears as a business front-end, and as such can be a bit off putting. There is but one direct mention of open-source, under a "What's in it for you" banner at the top.

Until you find the community portal for Ez Publish, it's all like a big storefront. Mentioning open-source and community in between selling itself. Indeed, running a \url{inurl:"http://ez.publish.no" "share.ez.no"} on google returns no results.

Poking around will let you find the open-source community web page\cite{ezcomweb}

\subsection{OpenDungeons}
\begin{quotation}
OpenDungeons is an open source, real time strategy game sharing game elements with the Dungeon Keeper series and Evil Genius. Players build an underground (or overground) dungeon which is inhabited by creatures. Players fight each other for control of territory by indirectly commanding their creatures, directly casting spells in combat, and luring enemies into sinister traps.\cite{odwebabout}
\end{quotation}


\section{Licenses}
\subsection{Media Player Classic}
MPC-HCs information about their licensing is easily found in the readme file in the source code\cite{mpcsource}. They are using the GPLv3\cite{gplv3}.
\subsection{Ez Publish}
Ez Publish uses a quinary licensing scheme\cite{ezpubweblicenses}. Most of them are for special use cases, meaning enterprises use the "eZ Business Use License Agreement" while most others use the GPLv2\cite{gplv2}. Information about their licensing and when to use which is more or less hidden.
\subsection{OpenDungeons}

\section{Infrastructure}
\subsection{Media Player Classic}
\subsubsection{Mailing Lists}
MPC-HC seems not to use mailing lists too much, it's hard to find on their page. It seems like their using irc channels instead. I did in the end find a mailing list, I had to search the page to find it. It's a mailing list for their bug report system, and is used as a notification to users to inform them about tickets in their bug reporting system\cite{mpctracsupport}\cite{mpctracmail}. This conflicts to how Karl Fogel is talking about the mailing lists in the book\cite{kfposs} of his. %It seems in the code that their referring to a development mailing list, maybe have to ask on irc.
\subsubsection{Version control}
MPC-HC's version control is easily find on the front page\cite{mpcsweb} under the news section. It's because they just recently changed it to GitHub. The changed where because it was easier for people to contribute, they say. This message about that they want more contributors goes through the entire page, and is also written much about in the readme\cite{mpcsource}.
\subsubsection{Bug tracker}
The bug tracker their using is the Trac\cite{tracweb} system, and is used to report bugs, feature request, or other kind of requests or changes. It is linked to several places on their page, and they has also made a guide%\cite{•} 
on how to proceed with a ticket ass they call a bug report, or any other report. 
\subsubsection{Instant messaging}
They are using irc channels\cite{mpccontact}, and seems like the main communication between the developers. Though, they only got two channels, one for users, help and such, and one for development channel. If you got some technical questions or something to discuss, it's here you do it. It's not the way Karl Fogel recommend doing things, because you have to be there when it happens to read it, and to learn from it. It also may be harder to get into the project if you're new to the project. On the other hand, the only got 168 000 lines of code\cite{ohlohmpc} at the moment, and have a pretty detailed to-do list\cite{mpctodo}.
\subsubsection{Website}
The website I would say is quite easy for a regular user to find what he need, download the software, report bugs, read about the. If you are more interested, or are contributing in the project, and spends some time in the development wiki\cite{mpcwiki} you easy find the things you want. Karl Fogel describes a website like this in his book:\begin{quotation}
(...)it act more as a glue for the other components than as a tool unto itself.\cite{kfposs}
\end{quotation} 
And that it was it does. From the page you can get all the information you need.
\subsection{Ez Publish}
\subsubsection{Mailing Lists}
Ez Publish has mailing lists\cite{ezpubwebmaillists}\footnote{At the first counseling Thomas stated that Ez Publish seemingly did not use mailing lists. It has been discovered that this was indeed false.} divided into 10 categories, half of which is bug related. All of the categories are more or less aimed at development/developers.
\subsubsection{Version control}
Ez Publish uses github\cite{ezpubgithub} as their community/open-source code interface/version control. Here they have lots of public repositories (40 repositories at the 30th of August 2012). They use primarily push requests to receive code from sources other than Ez Publish's own engineers. % this may be false, check mail.
\subsubsection{Bug tracker}
Ez Publish uses a bug tracking system from Waterproof Software\cite{waterproofweb} called wIT\cite{waterproofwebwit}. Hosted through the Ez publish platform.
\subsubsection{Instant messaging} % check in mail?
\subsubsection{Coding standards}
\subsubsection{Website}
All of Ez Publish's websites are hosted on the Ez Publish platform.
\subsubsection{Money}
Ez Publish straightforwardly presents itself as a company making money off of their support for the Ez Publish platform.

\subsection{OpenDungeons}
\subsubsection{Mailing Lists}
\subsubsection{Version control}
\subsubsection{Bug tracker}
\subsubsection{Instant messaging}
\subsubsection{Website}

\section{Conclusion}

\bibliographystyle{unsrt} % change for dcu, or something else entirely? unsrt is nice
\bibliography{ProjectComparisonRefs}
\end{document}
