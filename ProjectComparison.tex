\documentclass{report} % was article, thought i'd try with report...
\usepackage{float}
%\usepackage{fullpage} % 1 inch margins
%\usepackage{harvard} % harvard reference style++
\usepackage[hidelinks]{hyperref} % do not remember entirely what this does
%%\usepackage{minted} % [LINUXONLY] list environment with code highlighting

\title{Object oriented systems engineering \\ IMT3102 \\ Portfolio 1}
\author{Martin Kristian Mellum  \\ Thomas Sigurdsen 090320} % enter student number
\date{\today}

\begin{document}
\maketitle       % make title
%%\section{Abstract} % not sure if needed
%%This section is to be written when all else is done. 100-200 words summarizing.
\tableofcontents % make table of contents
\pagebreak	% and break the page, because pretty

\section{Introduction} % not sure if needed
In this first portfolio we will pick 3 - 4 open-source projects. We are going to analyze each of them from a software engineering point of view. Delve into the differences of infrastructure, communication, openness, financial and social aspects of the open-source development model for these projects. Much of this analysis will be done using chapter 3 of Karl Fogel's Producing Open Source Software\cite{kfposs} as a lense.

The projects we have initially chosen are:
\begin{itemize}
\item Media Player Classic\cite{mpcweb}
\item Ez publish\cite{ezpubweb}
\item OpenDungeons\cite{odweb} % I think we should drop this, pick a new better one, and write up a bit on the curiosa on this one.
\end{itemize}

\section{Project introduction}
\subsection{Media Player Classic}
\subsection{Ez Publish}
Ez Publish is a http based content management system. It is built using mainly php and effectively uses a dual licensing scheme.

Their about reads as follows:
\begin{quote}
eZ Systems is in the business of Web Content Management Solutions and has been since 1999. We maintain a global presence in Europe, Asia and the Americas. 

Our large international business partner network is the foundation of our success in the market. 

Our team of engineers, consultants and partner managers enables successful implementations for our customers and partners.\cite{ezpubwebabout}
\end{quote}
The Ez Publish website\cite{ezpubweb} appears as a business front-end, and as such can be a bit off putting. There is but one direct mention of open-source, under a "What's in it for you" banner at the top.

Until you find the community portal for Ez Publish, it's all like a big storefront. Mentioning open-source and community in between selling itself. Indeed, running a \url{inurl:"http://ez.publish.no" "share.ez.no"} on google returns no results. The open-source community web page\cite{ezcomweb} can be found through the Ez publish github\cite{ezpubgithub} or by doing searches with more keywords than simply "Ez publish".

Features are handled through a community roadmap, and an official roadmap. The official roadmap contains features that the product management at Ez has officially committed. As such these features see more of Ez's engineers than the community roadmap, which is the community's feature list.

\subsection{OpenDungeons}
\begin{quotation}
OpenDungeons is an open source, real time strategy game sharing game elements with the Dungeon Keeper series and Evil Genius. Players build an underground (or overground) dungeon which is inhabited by creatures. Players fight each other for control of territory by indirectly commanding their creatures, directly casting spells in combat, and luring enemies into sinister traps.\cite{odwebabout}
\end{quotation}


\section{Licenses}
\subsection{Media Player Classic}
\subsection{Ez Publish}
Ez Publish uses a quinary licensing scheme\cite{ezpubweblicenses}. Most of them are for special use cases, meaning enterprises use the "eZ Business Use License Agreement" while most others use the GPLv2\cite{gplv2}. Information about their licensing and when to use which is more or less hidden.
\subsection{OpenDungeons}

\section{Infrastructure}
\subsection{Media Player Classic}
\subsubsection{Mailing Lists}
\subsubsection{Version control}
\subsubsection{Bug tracker}
\subsubsection{Instant messaging}
\subsubsection{Website}

\subsection{Ez Publish}
\subsubsection{Mailing Lists}
Ez Publish has mailing lists\cite{ezpubwebmaillists}\footnote{At the first counseling Thomas stated that Ez Publish seemingly did not use mailing lists. It has been discovered that this was indeed false.} divided into 10 categories, half of which is bug related. All of the categories are more or less aimed at development and developers.
\subsubsection{Version control}
Ez Publish uses github\cite{ezpubgithub} as their community/open-source code interface/version control. Here they have lots of public repositories (40 repositories at the 30th of August 2012). They use primarily push requests to receive code from sources other than Ez Publish's own engineers. % this may be false, check mail.
\subsubsection{Bug tracker}
Ez Publish uses a bug tracking system from Waterproof Software\cite{waterproofweb} called wIT\cite{waterproofwebwit}. Hosted through the Ez publish platform.
\subsubsection{Instant messaging} % check in mail?
\subsubsection{Coding standards}
\subsubsection{Website}
All of Ez Publish's websites are hosted on the Ez Publish platform. The community portal\cite{ezcomweb} is fairly well written and structured. Although some information can be hard to find, this generally is not a problem.
\subsubsection{Money}
Ez Publish straightforwardly presents itself as a company making money off of their support for the Ez Publish platform. This is especially evident if you find their business portal\cite{ezpubweb} first.

\subsection{s}
\subsubsection{Mailing Lists}
\subsubsection{Version control}
\subsubsection{Bug tracker}
\subsubsection{Instant messaging}
\subsubsection{Website}

\section{Comparison}

\section{Conclusion} % not sure if needed?
\bibliographystyle{unsrt} % change for dcu, or something else entirely? unsrt is nice
\bibliography{ProjectComparisonRefs}
\end{document}
